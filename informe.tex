\documentclass[a4paper,11pt]{article}
\title{\textbf{}}
\author{}	
\date{}

\usepackage[spanish]{babel} % espanol
\usepackage{graphicx} % graficos
\usepackage{listings} % para el codigo
\usepackage{vmargin} % margenes 

\begin{document}

\begin{titlepage}
\begin{center}

\vspace*{-1in}
\begin{figure}[htb]
\begin{center}
\includegraphics[width=4cm]{logo.png}
\end{center}
\end{figure}
	
\begin{huge}
\textbf{Universidad de Buenos Aires\\
Facultad de Ingenier\'ia \\}
\vspace*{0.2in}
\end{huge}

\begin{LARGE}
\textbf{66.20 Organizaci\'on de computadoras \\}
\vspace*{1in}
\end{LARGE}

\begin{LARGE}
\textbf{Trabajo pr\'actico 0:
Infraestructura b\'asica \\
%\vspace*{0.2in}
}
\end{LARGE}
\begin{LARGE}
\textbf{1$^{er}$ cuatrimestre de 2015} \\
\end{LARGE}

\vspace*{1in}

\begin{Large}
\begin{tabular}{ l l l }
   Paula Saffioti & Padr\'on: 92001 & Email: paula.saffioti@gmail.com \\
   Kaoru Heanna & Padr\'on: 91891 & Email: kaoru.heanna@gmail.com \\
   Juli\'an Scialabba & Padr\'on: 92181 & Email: julian.scialabba@gmail.com \\
 \end{tabular}\\
\end{Large}

\end{center}
\end{titlepage}

\section{Documentaci\'on}
El objetivo del trabajo pr\'actico es desarrollar una versi\'on del comando tac de UNIX en lenguaje C. La funcionalidad b\'asica del mismo se basa en escribir por stdout el contenido de uno o m\'as archivos invirtiendo el orden de sus l\'ineas.

En la etapa de dise\~no analizamos de qu\'e manera ibamos a realizar el programa. Identificamos que deb\'iamos trabajar con archivos y memoria din\'amica y que la biblioteca standard de C nos prove\'ia la funcionalidad necesaria para estos requerimientos. Luego, pensamos con un mayor nivel de detalle el algoritmo base del programa que se resume en la siguiente lista de pasos:
\begin{enumerate}
	\item Abrir un archivo pasado por par\'ametro.
	\item Leer el archivo l\'inea por l\'inea hasta el final del archivo y guardarlo en un array. El tama\~no de las l\'ineas debe ser din\'amico.
	\item Recorrer el array en sentido inverso imprimiendo por stdout el contenido de cada l\'inea.
\end{enumerate}

El paso 1 consiste en abrir los archivos pasados por par\'ametro. Esto lo resolvimos f\'acilmente con la funci\'on fopen. Como resultado de este paso, nos qued\'o un puntero al archivo para operar con \'el. En caso de alg\'un error, lo lanzamos por stderr.

El paso 2, sin dudas, es el m\'as complejo del trabajo pr\'actico. La complejidad reside en la lectura del archivo pidiendo memoria de manera din\'amica. 
Identificamos que para leer l\'inea por l\'inea el archivo podiamos hacerlo mediante la lectura de caracter por caracter hasta un fin de l\'inea o la lectura de un bloque de caracteres hasta llegar al fin de l\'inea (fget vs fgets). Hemos optado por la segunda opci\'on por temas de rendimiento y practicidad.
Luego de elegir leer el archivo mediante fgets, observamos que dicha funci\'on le\'ia una cantidad fija de caracteres por l\'inea y que los archivos con los que \'ibamos a trabajar esta cantidad iba a ser din\'amica. Claramente \'ibamos a tener que utilizar los m\'etodos de manejo de memoria de c: malloc, calloc, realloc y free.

En detalle de implementaci\'on, esto lo resolvimos de la siguiente manera:
\begin{enumerate}
	\item Por cada nueva l\'inea, inicializamos un array de caracteres que representaba la l\'inea leer. La asignaci\'on de memoria din\'amica a este array de caracteres lo hicimos mediante la funciona calloc que adem\'as de reservar la memoria estipulada la inicializa en 0.
	\item Leemos bloques de caracteres mediante fgets en un buffer y lo copiamos en el array de caracteres que representa la l\'inea actual mencionada en el paso 1. En esta copia, tuvimos que redimensionar el array de caracteres mediante un realloc.
	\item Si el \'ultimo caracter del buffer era un fin de l\'inea, se proced\'ia a guardar la l\'inea actual resultante en el array de l\'ineas y se inicializaba un nuevo array de caracteres como en el paso 1.
\end{enumerate}


El paso 3. fue el m\'as simple, ya que con un solo recorrido del array formado anteriormente lo cumplimos sin problemas.

\section{Comandos}
Para compilar el programa es necesario escribir el siguiente comando en el directorio donde se encuentre main.c:

\lstset{language=bash, breaklines=true, basicstyle=\normalsize}
\begin{lstlisting}
$ gcc -o tp0 main.c
\end{lstlisting}

\section{Corridas de prueba}
Para corroborar el funcionamiento del programa cont\'abamos con una serie de archivos de prueba y con el diccionario de palabras ubicado en /usr/share/dict/words. 
\\
\\
\underline{Salidas de ejemplo:}\\\\
\textbf{return.txt:}\\
\emph{of -1 is returned and errno is set to indicate the error.
Upon successful completion a value of 0 is returned. Otherwise, a value}
\\\\
\textbf{basic.txt}\\
\emph{4. D\\
3. C\\
2. B\\
1. A\\}\\
\textbf{return.txt basic.txt}\\
\emph{of -1 is returned and errno is set to indicate the error.\\
Upon successful completion a value of 0 is returned. Otherwise, a value\\
4. D\\
3. C\\
2. B\\
1. A}\\

Para poder hacer todas las pruebas de manera r\'apida escribimos un script que compilaba el programa, luego lo ejecutaba para los archivos mencionados d\'andolos vuelta dos veces y finalmente verificaba que el archivo resultante fuera igual al original con md5sum.\\ 

\underline{MD5 de los archivos de prueba procesados dos veces:}\\\\
\textbf{empty.txt:} d41d8cd98f00b204e9800998ecf8427e\\
\textbf{basic:} f82584604433de83165f0227f4f06c2c\\
\textbf{empty-lines.txt:} 0c060d8e86deedcb0f9cd89f6a3f9c93\\
\textbf{large-file.txt:} 8b2a59a337c384be9b640a38a377eef1\\
\textbf{return.txt:} 6777acdeb99f2fb0ed9e14fc82899e82\\
\textbf{status.txt:} 964ccaae773b2f9044dc549ed3636ab6\\
\textbf{words:} cbbcded3dc3b61ad50c4e36f79c37084\\


Para el caso de NetBSD, probamos el funcionamiento del programa para todos los archivos a mano. 


\section{C\'odigo fuente}

\lstset{language=C, breaklines=true, basicstyle=\normalsize}
\begin{lstlisting}
#include <stdio.h>
#include <stdlib.h>
#include <string.h>

void printHelpInfo() {
    printf("%s", "Usage:\n");
    printf("%s", "	tp0 -h\n");
    printf("%s", "	tp0 -V\n");
    printf("%s", "	tp0 [file...]\n");
    printf("%s", "Options:\n");
    printf("%s", "	-V, --version Print version and quit.\n");
    printf("%s", "	-h, --help Print this information and quit.\n");
    printf("%s", "Examples:\n");
    printf("%s", "	tp0 foo.txt bar.txt\n");
    printf("%s", "	tp0 gz.txt\n");
}

void printVersionInfo() {
    printf("%s", "tp0 1.1\n");
    printf("%s", "Copyright © 2015 FIUBA.\n");
    printf("%s", "Esto es software libre: usted es libre de cambiarlo y redistribuirlo.\n");
    printf("%s", "No hay NINGUNA GARANTÍA, hasta donde permite la ley.\n");
    printf("%s", "\n");
    printf("%s", "Escrito por Julian Scialabba, Kaoru Heanna y Paula Saffioti.\n");
}

int isEndOfLine(char c) {
    if (c == '\n') {
        return 1;
    }
    return 0;
}

void concatBuffer(char** line, const char* buffer) {
    size_t len1 = *line ? strlen(*line) : 0;
    size_t len2 = buffer ? strlen(buffer) : 0;
    char* concat = realloc(*line, len1 + len2 + 1);
    if (concat) {
        memcpy(concat + len1, buffer, len2 + 1);
        *line = concat;
    }
}

int tacFile(FILE* fp) {
    const int bufIncrSize = 10;
    char ** arrayLines = NULL;
    int lineCounter = 0;
    char buffer[bufIncrSize];
    char* line = (char *) calloc(bufIncrSize, sizeof (char));
    while (fgets(buffer, bufIncrSize, fp)) {
        concatBuffer(&line, buffer);
        char lastCharacterBuffer = buffer[strlen(buffer) - 1];
        if (isEndOfLine(lastCharacterBuffer)) {
            arrayLines = realloc(arrayLines,
                    (lineCounter + 1) * sizeof (char *));
            arrayLines[lineCounter] = line;
            lineCounter++;
            line = (char *) calloc(bufIncrSize, sizeof (char));
        }
    }

    int i;
    for (i = lineCounter - 1; i >= 0; i--) {
        printf("%s", arrayLines[i]);
        free(arrayLines[i]);

    }
    free(line);
    free(arrayLines);
    return (EXIT_SUCCESS);
}

int main(int argc, char** argv) {

    if ((argc == 2) && ((strcmp(argv[1], "-h") == 0) || (strcmp(argv[1], "--help") == 0))) {
        printHelpInfo();
        return (EXIT_SUCCESS);
    }

    if ((argc == 2) && ((strcmp(argv[1], "-V") == 0) || (strcmp(argv[1], "--version") == 0))) {
        printVersionInfo();
        return (EXIT_SUCCESS);
    }

    int result;

    if (argc < 2) { //no tengo archivo de entrada, uso standard input
        result = tacFile(stdin);
        return (result);
    }

    int i;
    for (i = 1; i < argc; i++) {
        FILE *fp;
        fp = fopen(argv[i], "r");
        if (fp == NULL) {
            fprintf(stderr, "%s", argv[i]);
            fprintf(stderr, ": nombre de archivo o comando inválido.\n");
            return (EXIT_FAILURE);
        }
        tacFile(fp);
        fclose(fp);
    }
    return (EXIT_SUCCESS);
}

\end{lstlisting}

\section{C\'odigo MIPS32}
\lstset{breaklines=true, basicstyle=\normalsize}
\begin{lstlisting}
	.file	1 "main.c"
	.section .mdebug.abi32
	.previous
	.abicalls
	.rdata
	.align	2
$LC0:
	.ascii	"%s\000"
	.align	2
$LC1:
	.ascii	"Usage:\n\000"
	.align	2
$LC2:
	.ascii	"\ttp0 -h\n\000"
	.align	2
$LC3:
	.ascii	"\ttp0 -V\n\000"
	.align	2
$LC4:
	.ascii	"\ttp0 [file...]\n\000"
	.align	2
$LC5:
	.ascii	"Options:\n\000"
	.align	2
$LC6:
	.ascii	"\t-V, --version Print version and quit.\n\000"
	.align	2
$LC7:
	.ascii	"\t-h, --help Print this information and quit.\n\000"
	.align	2
$LC8:
	.ascii	"Examples:\n\000"
	.align	2
$LC9:
	.ascii	"\ttp0 foo.txt bar.txt\n\000"
	.align	2
$LC10:
	.ascii	"\ttp0 gz.txt\n\000"
	.text
	.align	2
	.globl	printHelpInfo
	.ent	printHelpInfo
printHelpInfo:
	.frame	$fp,40,$ra		# vars= 0, regs= 3/0, args= 16, extra= 8
	.mask	0xd0000000,-8
	.fmask	0x00000000,0
	.set	noreorder
	.cpload	$t9
	.set	reorder
	subu	$sp,$sp,40
	.cprestore 16
	sw	$ra,32($sp)
	sw	$fp,28($sp)
	sw	$gp,24($sp)
	move	$fp,$sp
	la	$a0,$LC0
	la	$a1,$LC1
	la	$t9,printf
	jal	$ra,$t9
	la	$a0,$LC0
	la	$a1,$LC2
	la	$t9,printf
	jal	$ra,$t9
	la	$a0,$LC0
	la	$a1,$LC3
	la	$t9,printf
	jal	$ra,$t9
	la	$a0,$LC0
	la	$a1,$LC4
	la	$t9,printf
	jal	$ra,$t9
	la	$a0,$LC0
	la	$a1,$LC5
	la	$t9,printf
	jal	$ra,$t9
	la	$a0,$LC0
	la	$a1,$LC6
	la	$t9,printf
	jal	$ra,$t9
	la	$a0,$LC0
	la	$a1,$LC7
	la	$t9,printf
	jal	$ra,$t9
	la	$a0,$LC0
	la	$a1,$LC8
	la	$t9,printf
	jal	$ra,$t9
	la	$a0,$LC0
	la	$a1,$LC9
	la	$t9,printf
	jal	$ra,$t9
	la	$a0,$LC0
	la	$a1,$LC10
	la	$t9,printf
	jal	$ra,$t9
	move	$sp,$fp
	lw	$ra,32($sp)
	lw	$fp,28($sp)
	addu	$sp,$sp,40
	j	$ra
	.end	printHelpInfo
	.size	printHelpInfo, .-printHelpInfo
	.rdata
	.align	2
$LC11:
	.ascii	"tp0 1.1\n\000"
	.align	2
$LC12:
	.ascii	"Copyright \302\251 2015 FIUBA.\n\000"
	.align	2
$LC13:
	.ascii	"Esto es software libre: usted es libre de cambiarlo y re"
	.ascii	"distribuirlo.\n\000"
	.align	2
$LC14:
	.ascii	"No hay NINGUNA GARANT\303\215A, hasta donde permite la l"
	.ascii	"ey.\n\000"
	.align	2
$LC15:
	.ascii	"\n\000"
	.align	2
$LC16:
	.ascii	"Escrito por Julian Scialabba, Kaoru Heanna y Paula Saffi"
	.ascii	"oti.\n\000"
	.text
	.align	2
	.globl	printVersionInfo
	.ent	printVersionInfo
printVersionInfo:
	.frame	$fp,40,$ra		# vars= 0, regs= 3/0, args= 16, extra= 8
	.mask	0xd0000000,-8
	.fmask	0x00000000,0
	.set	noreorder
	.cpload	$t9
	.set	reorder
	subu	$sp,$sp,40
	.cprestore 16
	sw	$ra,32($sp)
	sw	$fp,28($sp)
	sw	$gp,24($sp)
	move	$fp,$sp
	la	$a0,$LC0
	la	$a1,$LC11
	la	$t9,printf
	jal	$ra,$t9
	la	$a0,$LC0
	la	$a1,$LC12
	la	$t9,printf
	jal	$ra,$t9
	la	$a0,$LC0
	la	$a1,$LC13
	la	$t9,printf
	jal	$ra,$t9
	la	$a0,$LC0
	la	$a1,$LC14
	la	$t9,printf
	jal	$ra,$t9
	la	$a0,$LC0
	la	$a1,$LC15
	la	$t9,printf
	jal	$ra,$t9
	la	$a0,$LC0
	la	$a1,$LC16
	la	$t9,printf
	jal	$ra,$t9
	move	$sp,$fp
	lw	$ra,32($sp)
	lw	$fp,28($sp)
	addu	$sp,$sp,40
	j	$ra
	.end	printVersionInfo
	.size	printVersionInfo, .-printVersionInfo
	.align	2
	.globl	isEndOfLine
	.ent	isEndOfLine
isEndOfLine:
	.frame	$fp,24,$ra		# vars= 8, regs= 2/0, args= 0, extra= 8
	.mask	0x50000000,-4
	.fmask	0x00000000,0
	.set	noreorder
	.cpload	$t9
	.set	reorder
	subu	$sp,$sp,24
	.cprestore 0
	sw	$fp,20($sp)
	sw	$gp,16($sp)
	move	$fp,$sp
	move	$v0,$a0
	sb	$v0,8($fp)
	lb	$v1,8($fp)
	li	$v0,10			# 0xa
	bne	$v1,$v0,$L20
	li	$v0,1			# 0x1
	sw	$v0,12($fp)
	b	$L19
$L20:
	sw	$zero,12($fp)
$L19:
	lw	$v0,12($fp)
	move	$sp,$fp
	lw	$fp,20($sp)
	addu	$sp,$sp,24
	j	$ra
	.end	isEndOfLine
	.size	isEndOfLine, .-isEndOfLine
	.align	2
	.globl	concatBuffer
	.ent	concatBuffer
concatBuffer:
	.frame	$fp,64,$ra		# vars= 24, regs= 3/0, args= 16, extra= 8
	.mask	0xd0000000,-8
	.fmask	0x00000000,0
	.set	noreorder
	.cpload	$t9
	.set	reorder
	subu	$sp,$sp,64
	.cprestore 16
	sw	$ra,56($sp)
	sw	$fp,52($sp)
	sw	$gp,48($sp)
	move	$fp,$sp
	sw	$a0,64($fp)
	sw	$a1,68($fp)
	lw	$v0,64($fp)
	lw	$v0,0($v0)
	beq	$v0,$zero,$L22
	lw	$v0,64($fp)
	lw	$a0,0($v0)
	la	$t9,strlen
	jal	$ra,$t9
	sw	$v0,36($fp)
	b	$L23
$L22:
	sw	$zero,36($fp)
$L23:
	lw	$v0,36($fp)
	sw	$v0,24($fp)
	lw	$v0,68($fp)
	beq	$v0,$zero,$L24
	lw	$a0,68($fp)
	la	$t9,strlen
	jal	$ra,$t9
	sw	$v0,40($fp)
	b	$L25
$L24:
	sw	$zero,40($fp)
$L25:
	lw	$v0,40($fp)
	sw	$v0,28($fp)
	lw	$a0,64($fp)
	lw	$v1,24($fp)
	lw	$v0,28($fp)
	addu	$v0,$v1,$v0
	addu	$v0,$v0,1
	lw	$a0,0($a0)
	move	$a1,$v0
	la	$t9,realloc
	jal	$ra,$t9
	sw	$v0,32($fp)
	lw	$v0,32($fp)
	beq	$v0,$zero,$L21
	lw	$v1,32($fp)
	lw	$v0,24($fp)
	addu	$v1,$v1,$v0
	lw	$v0,28($fp)
	addu	$v0,$v0,1
	move	$a0,$v1
	lw	$a1,68($fp)
	move	$a2,$v0
	la	$t9,memcpy
	jal	$ra,$t9
	lw	$v1,64($fp)
	lw	$v0,32($fp)
	sw	$v0,0($v1)
$L21:
	move	$sp,$fp
	lw	$ra,56($sp)
	lw	$fp,52($sp)
	addu	$sp,$sp,64
	j	$ra
	.end	concatBuffer
	.size	concatBuffer, .-concatBuffer
	.align	2
	.globl	tacFile
	.ent	tacFile
tacFile:
	.frame	$fp,72,$ra		# vars= 32, regs= 3/0, args= 16, extra= 8
	.mask	0xd0000000,-8
	.fmask	0x00000000,0
	.set	noreorder
	.cpload	$t9
	.set	reorder
	subu	$sp,$sp,72
	.cprestore 16
	sw	$ra,64($sp)
	sw	$fp,60($sp)
	sw	$gp,56($sp)
	move	$fp,$sp
	sw	$a0,72($fp)
	sw	$sp,48($fp)
	li	$v0,10			# 0xa
	sw	$v0,24($fp)
	sw	$zero,28($fp)
	sw	$zero,32($fp)
	lw	$v0,24($fp)
	addu	$v0,$v0,7
	srl	$v0,$v0,3
	sll	$v0,$v0,3
	subu	$sp,$sp,$v0
	addu	$v0,$sp,16
	sw	$v0,52($fp)
	lw	$a0,24($fp)
	li	$a1,1			# 0x1
	la	$t9,calloc
	jal	$ra,$t9
	sw	$v0,36($fp)
$L28:
	lw	$a0,52($fp)
	lw	$a1,24($fp)
	lw	$a2,72($fp)
	la	$t9,fgets
	jal	$ra,$t9
	bne	$v0,$zero,$L30
	b	$L29
$L30:
	addu	$v0,$fp,36
	move	$a0,$v0
	lw	$a1,52($fp)
	la	$t9,concatBuffer
	jal	$ra,$t9
	lw	$a0,52($fp)
	la	$t9,strlen
	jal	$ra,$t9
	lw	$v1,52($fp)
	addu	$v0,$v0,$v1
	addu	$v0,$v0,-1
	lbu	$v0,0($v0)
	sb	$v0,40($fp)
	lb	$v0,40($fp)
	move	$a0,$v0
	la	$t9,isEndOfLine
	jal	$ra,$t9
	beq	$v0,$zero,$L28
	lw	$v0,32($fp)
	sll	$v0,$v0,2
	addu	$v0,$v0,4
	lw	$a0,28($fp)
	move	$a1,$v0
	la	$t9,realloc
	jal	$ra,$t9
	sw	$v0,28($fp)
	lw	$v0,32($fp)
	sll	$v1,$v0,2
	lw	$v0,28($fp)
	addu	$v1,$v1,$v0
	lw	$v0,36($fp)
	sw	$v0,0($v1)
	lw	$v0,32($fp)
	addu	$v0,$v0,1
	sw	$v0,32($fp)
	lw	$a0,24($fp)
	li	$a1,1			# 0x1
	la	$t9,calloc
	jal	$ra,$t9
	sw	$v0,36($fp)
	b	$L28
$L29:
	lw	$v0,32($fp)
	addu	$v0,$v0,-1
	sw	$v0,44($fp)
$L32:
	lw	$v0,44($fp)
	bgez	$v0,$L35
	b	$L33
$L35:
	lw	$v0,44($fp)
	sll	$v1,$v0,2
	lw	$v0,28($fp)
	addu	$v0,$v1,$v0
	la	$a0,$LC0
	lw	$a1,0($v0)
	la	$t9,printf
	jal	$ra,$t9
	lw	$v0,44($fp)
	sll	$v1,$v0,2
	lw	$v0,28($fp)
	addu	$v0,$v1,$v0
	lw	$a0,0($v0)
	la	$t9,free
	jal	$ra,$t9
	lw	$v0,44($fp)
	addu	$v0,$v0,-1
	sw	$v0,44($fp)
	b	$L32
$L33:
	lw	$a0,36($fp)
	la	$t9,free
	jal	$ra,$t9
	lw	$a0,28($fp)
	la	$t9,free
	jal	$ra,$t9
	lw	$sp,48($fp)
	move	$v0,$zero
	move	$sp,$fp
	lw	$ra,64($sp)
	lw	$fp,60($sp)
	addu	$sp,$sp,72
	j	$ra
	.end	tacFile
	.size	tacFile, .-tacFile
	.rdata
	.align	2
$LC17:
	.ascii	"-h\000"
	.align	2
$LC18:
	.ascii	"--help\000"
	.align	2
$LC19:
	.ascii	"-V\000"
	.align	2
$LC20:
	.ascii	"--version\000"
	.align	2
$LC21:
	.ascii	"r\000"
	.align	2
$LC22:
	.ascii	": nombre de archivo o comando inv\303\241lido.\n\000"
	.text
	.align	2
	.globl	main
	.ent	main
main:
	.frame	$fp,56,$ra		# vars= 16, regs= 3/0, args= 16, extra= 8
	.mask	0xd0000000,-8
	.fmask	0x00000000,0
	.set	noreorder
	.cpload	$t9
	.set	reorder
	subu	$sp,$sp,56
	.cprestore 16
	sw	$ra,48($sp)
	sw	$fp,44($sp)
	sw	$gp,40($sp)
	move	$fp,$sp
	sw	$a0,56($fp)
	sw	$a1,60($fp)
	lw	$v1,56($fp)
	li	$v0,2			# 0x2
	bne	$v1,$v0,$L37
	lw	$v0,60($fp)
	addu	$v0,$v0,4
	lw	$a0,0($v0)
	la	$a1,$LC17
	la	$t9,strcmp
	jal	$ra,$t9
	beq	$v0,$zero,$L38
	lw	$v0,60($fp)
	addu	$v0,$v0,4
	lw	$a0,0($v0)
	la	$a1,$LC18
	la	$t9,strcmp
	jal	$ra,$t9
	bne	$v0,$zero,$L37
$L38:
	la	$t9,printHelpInfo
	jal	$ra,$t9
	sw	$zero,36($fp)
	b	$L36
$L37:
	lw	$v1,56($fp)
	li	$v0,2			# 0x2
	bne	$v1,$v0,$L39
	lw	$v0,60($fp)
	addu	$v0,$v0,4
	lw	$a0,0($v0)
	la	$a1,$LC19
	la	$t9,strcmp
	jal	$ra,$t9
	beq	$v0,$zero,$L40
	lw	$v0,60($fp)
	addu	$v0,$v0,4
	lw	$a0,0($v0)
	la	$a1,$LC20
	la	$t9,strcmp
	jal	$ra,$t9
	bne	$v0,$zero,$L39
$L40:
	la	$t9,printVersionInfo
	jal	$ra,$t9
	sw	$zero,36($fp)
	b	$L36
$L39:
	lw	$v0,56($fp)
	slt	$v0,$v0,2
	beq	$v0,$zero,$L41
	la	$a0,__sF
	la	$t9,tacFile
	jal	$ra,$t9
	sw	$v0,24($fp)
	lw	$v0,24($fp)
	sw	$v0,36($fp)
	b	$L36
$L41:
	li	$v0,1			# 0x1
	sw	$v0,28($fp)
$L42:
	lw	$v0,28($fp)
	lw	$v1,56($fp)
	slt	$v0,$v0,$v1
	bne	$v0,$zero,$L45
	b	$L43
$L45:
	lw	$v0,28($fp)
	sll	$v1,$v0,2
	lw	$v0,60($fp)
	addu	$v0,$v1,$v0
	lw	$a0,0($v0)
	la	$a1,$LC21
	la	$t9,fopen
	jal	$ra,$t9
	sw	$v0,32($fp)
	lw	$v0,32($fp)
	bne	$v0,$zero,$L46
	lw	$v0,28($fp)
	sll	$v1,$v0,2
	lw	$v0,60($fp)
	addu	$v0,$v1,$v0
	la	$a0,__sF+176
	la	$a1,$LC0
	lw	$a2,0($v0)
	la	$t9,fprintf
	jal	$ra,$t9
	la	$a0,__sF+176
	la	$a1,$LC22
	la	$t9,fprintf
	jal	$ra,$t9
	li	$v0,1			# 0x1
	sw	$v0,36($fp)
	b	$L36
$L46:
	lw	$a0,32($fp)
	la	$t9,tacFile
	jal	$ra,$t9
	lw	$a0,32($fp)
	la	$t9,fclose
	jal	$ra,$t9
	lw	$v0,28($fp)
	addu	$v0,$v0,1
	sw	$v0,28($fp)
	b	$L42
$L43:
	sw	$zero,36($fp)
$L36:
	lw	$v0,36($fp)
	move	$sp,$fp
	lw	$ra,48($sp)
	lw	$fp,44($sp)
	addu	$sp,$sp,56
	j	$ra
	.end	main
	.size	main, .-main
	.ident	"GCC: (GNU) 3.3.3 (NetBSD nb3 20040520)"
\end{lstlisting}

\end{document}
